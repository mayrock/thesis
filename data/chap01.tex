\chapter{绪论}
\label{cha:intro}

\section{研究背景}
计算机技术问世之初,计算设备的最主要功能是单纯进行大量的复杂计算工作;随着技术的发展,特别是个人计算机以及互联网的普及,计算设备的功能也扩展到生产力工具和通讯、信息获取工作。近几年来,随着移动计算设备的蓬勃发展\cite{forman1994challenges}\cite{satyanarayanan2011mobile},人们使用计算设备的场景也发生了翻天覆地的变化。智能手机、平板电脑等移动计算设备越来越多地参与到通讯、社交、娱乐、工作等日常生活的各个方面,自带设备工作(Bring Your Own Device,BYOD)潮流的流行,使得计算设备从单纯用于计算,发展成了与用户日常行为联系非常紧密的个人随身物品。
这些计算设备都具有贴身携带、实时联网、传感器丰富等特点,能够在设备端及移动运营商网络中产生大量的用户行为数据。
这些用户行为数据忠实地记录了用户日常行为的点点滴滴,从而给用户行为模式的量化客观分析提供了可能\cite{mascolo2011mobile}\cite{laurila2012mobile}。
移动环境下的用户行为分析在计算机以及其他多个领域都有着广泛的应用。移动环境下的用户行为模型能够用于用户内容推荐,移动运营商网络优化,聚合的群体行为模式分析,以及动态社会学分析等。


使用计算机方法进行用户行为分析大多需要大量记录用户行为的日志数据。传统的方法主要使用用户的互联网使用记录,来进行互联网中的用户上网行为分析\cite{liu2008identifying}\cite{granka2004eye}。 移动环境下,用户行为分析也发展到包含上网行为,设备使用行为,以及移动设备能够记录或反映的其他日常生活行为等等。移动环境下的用户行为分析具有数据来源种类繁多,关系复杂,数据量巨大,以及挖掘内容多样等特点,这也给该问题的研究带来了挑战。目前,传统的用户行为分析方法无法有效地将时间、空间、用户、社会关系、设备使用情况等诸多数据进行有效的整合和分析,难以快速提供紧密联系用户生活的分析结果。一些研究者针对这些问题提出了不同的模型分析方法\cite{www08cui}\cite{ist06tseng}\cite{infocom06kim},但都在数据来源、用户规模以及挖掘结果的应用范围等方面存在不同的缺陷。

随身携带的移动设备的迅猛发展使得很多以前看似与计算机领域无关的人类行为学、心理学等领域的难题可以运用基于移动数据的用户行为分析方法来实现。如心境评估问题。心境(英文为Mood)及心境障碍问题是现代社会高压力下人们面临的最普遍的心理问题之一。心境障碍的干预和治疗离不开及时、准确的心境状态评估。但是,由于心境的极度主观性和不一致性,使得心境评估成为当前心理学领域的难点问题之一。目前常见的心境评估方法主要调查问卷和直接进行心理咨询等传统的心理测量学方法。这些方法有很大的局限性。首先,操作的复杂导致每次评估过程极其不方便,影响了评估进行的频率,加之心境的不稳定性,相对静态、非实时的传统方法不能记录每天的心境变化情况,给心境的评估效果带来了很大的影响。其次,评估过程需要耗费大量的精力,也限制了心境评估评估的应用场景。最后,人们对于自身心境状态有着本能的隐藏倾向,会不自觉地在各种评估方法中隐瞒自己的真实心境状态,这使得设计准确度高的心境评估量表十分重要,进而加大了心境评估的复杂程度,而且限制了传统方法的评估效果。

另外,日益增长的数据量、数据种类和数据分析需求也给数据分析的工具和平台提出了新的要求。一方面,要充分利用各种并行化存储和计算方法;另一方面要能灵活地对多种来源的数据进行分析和集成。单纯使用Hadoop\cite{borthakur2007hadoop}或MPI\cite{gropp1999using}等并行化解决方案中的某一种并不能完全满足不同数据和不同挖掘任务的需求,更多的并行化平台也开始涌现\cite{engle2012shark}。同时,如何对多种数据集进行统一的管理和浏览、可视化,使得数据挖掘研究者和数据分析人员能够不必关心编程细节,而专注于数据本身,这也是平台设计上亟待解决的问题。

\section{论文的内容和主要贡献}
基于上述背景,本文提出基于移动数据对用户行为进行分析挖掘的完整架构,包括数据的收集,挖掘算法设计,并行挖掘平台的搭建,以及挖掘结果可视化展示等,能够实现针对用户个人以及群体行为模式进行分析挖掘的完整流程。

\subsection{个体用户行为分析方法}
在面向个体用户的行为分析方面,针对现代社会人们普遍压力较大,心理健康问题(特别是心境障碍问题)日益严峻的现状,本文提出了使用手机传感器及通讯数据,应用用户行为分析手段来评估用户心境的方法。

本文将智能手机感知数据(包括传感器数据,通讯记录以及手机使用数据等),结合用户的移动社交网络信息,来解析出用户日常行为特征,如肢体运动,微动作,位置轨迹等等。而后使用基于因子图模型的机器学习方法,将这些特征结合起来,对用户每日的心境进行实时、客观地检测和评估。同时,将提出的方法通过450人·天的实验予以验证,得到了较好的模型表现。

相较于传统的基于问卷的自报告评估方法,本文提出的基于用户行为分析心境评估方法克服了其主观性,不一致性和高延时的问题。手机收集的数据能够保证真实可靠,不会撒谎,克服了人思维的欺骗性;同时,手机检测的被动性,使得整个评估过程用户干预达到最小;手机的随身携带和随时开机,也保证了心境评估的及时性。这些优点使得本方法可以在心理治疗、心境障碍预警和干预等方面发挥作用。

\subsection{群体用户行为分析方法}
在面向服务提供商的群体用户行为分析方面,针对移动互联网用户上网兴趣和惯常模式,本文提出了使用移动运营商数据分析用户群体上网兴趣和模式发现的方法。将用户上网行为中的地理位置、用户兴趣、网站类型等因素进行统一建模和分析,以分析用户上网模式,预测上网行为。

由于用户在不同的功能区域中上网行为习惯不同,本文提出了结合地理位置信息和不同位置下移动互联网的使用情况,对城市进行功能区域划分的方法;利用概率话题模型,将地理功能区域、用户和网站进行统一的建模,得出根据用户重合度聚类的网站类簇(用户兴趣),以及不同用户在不同地理功能区域上的兴趣分布。

本研究中提出了一个新的地理区域划分的方法,结合了地理特征和移动互联网使用记录。通过聚类得到的地理功能区域拥有更多的语义信息,从而可以为用户行为分析提供更好的上下文。该方法应用基于密度的聚类方法,结合点(基站位置)的地理距离、用户上网记录分布的相似度以及地点间迁移上网记录数等条件来将点划分为地理功能区域。

同时,本研究提出了处理原始的用户互联网访问记录,将主机地址(URL)按照所属的网站(服务)进行聚类的方法。根据主机地址之间的用户访问的重合度以及地址相似性来从原始HTTP请求的地址中发现网站(服务)。

本研究中提出了一个新颖的用户的移动互联网使用行为分析的概率模型,从真实的移动运营商数据上的实验表明,该模型在模式分析和行为还原(预测)上均有良好的表现。提出的模型可以用来做城市级别的聚合行为分析,也可用来做移动互联网中使用情况的预测和服务的推荐。


\subsection{移动环境下用户行为分析平台}
针对移动环境下数据种类多,规模大等特征,本文提出并建立了并行化的移动用户行为分析平台,包括与具体存储无关的数据集操控API,REST接口以及数据可视化展示等,实现了对不同来源的数据进行统一的集成、处理、分析挖掘和可视化。

平台设计实现了以数据集为操作对象的API接口,对不同来源、不同原始结构的数据进行统一管理,并且通过接口支持不同的数据存储实现,对平台用户来说屏蔽了底层数据存储的实现细节,保证浏览、读取和操作数据集简单方便。

平台包含一整套用于查询、遍历数据的API,支持按记录过滤\textit{where}、按列过滤\textit{select}、按某些列进行分组统计、排序和内连接等操作。查询API在与底层数据存储实现完全无关的条件下,实现了对数据灵活的查询和获取,保证了整个平台对不同数据、不同存储方式的一致性,也是数据集之间灵活的集成、算法和灵活迁移成为可能。

平台还提供了对数据进行可视化的通用模块。根据数据集自身的语义信息,利用多种可视化手段(地图,网络结构图,饼状、折线、柱状统计图等),对数据集进行通用而不失针对性的可视化展示,使得数据的统计特征和挖掘结果能够更好的展现出来。

\section{论文的主要结构}

第\ref{cha:mood}章提出了使用手机端数据,应用用户行为分析手段进行心境评估的方法。第\ref{mood:relatedwork}章简要介绍了基于手机监测数据的用户行为建模,以及心境评估方面的相关工作;第\ref{mood:data}章对实验数据的基本情况及其各方面特征进行了简要描述,并据此提出了进行心境评估的因子图模型;第\ref{mood:evaluation}章展示了实验结果及分析;第\ref{mood:conclusion}章总结了本章。

第\ref{cha:interest}章描述了根据移动运营商的用户上网记录进行用户在移动互联网上的上网行为分析的方法和分析结果。第\ref{interest:sec:relatedwork}章描述了移动互联网上的用户行为分析以及基于位置的分析的相关工作;第\ref{interest:sec:data}章对研究所用的数据集总体描述,以及从多角度对数据进行了观察并展示了观察结果;第\ref{interest:sec:region}章描述了从原始HTTP记录中根据用户上网行为来进行地理区域发现的方法和结果;第\ref{interest:sec:uri}章详细描述了用来进行移动互联网内的用户行为建模的概率话题模型; 第\ref{interest:sec:exp}章描述了将多种模型应用于数据集的实验结果;第\ref{interest:sec:discussion}章给出了针对模型表现和实验结果的详细讨论,从不同角度分析了实验结果,并给出了模型的进一步应用方向;第\ref{interest:sec:conclusion}章为第\ref{cha:interest}章总结。

第\ref{cha:system}章描述了用于移动环境下的用户行为分析的\textbf{移动用户行为分析平台}。第\ref{system:sec:intro}章给出了平台结构和功能的概述;第\ref{system:sec:collect}章描述了Android系统上的手机端的数据采集系统实现,包括数据采集的类型、传感器数据采集策略、传感器能耗管理策略等;第\ref{system:sec:mdap}章描述了后端的数据分析平台,包括其编程接口的设计与实现,以及实现远程访问的REST API的设计与实现;第\ref{system:sec:vis}章描述了平台数据可视化模块。第\ref{system:sec:conclusion}对第\ref{cha:system}章进行了小结。

最后,第\ref{cha:conclusion}章对全文进行了总结,并提出了进一步工作的方向。

