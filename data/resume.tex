\begin{resume}

  \resumeitem{个人简历}

  1989 年 5 月 26 日出生于辽宁省盖州市。
  
  2006 年 9 月考入北京工商大学大学计算机科学与技术系,2010 年 7 月本科毕业并获得工学学士学位。
  
  2010 年 9 月通过研究生入学统一考试进入清华大学计算机科学与技术系攻读硕士学位至今。

  \resumeitem{发表的学术论文} % 发表的和录用的合在一起

  \begin{enumerate}[{[}1{]}]
  \item Ma Y, Xu B, Bai Y, et al. Building linked open university data: tsinghua university open data
as a showcase. volume 7185 LNCS, Hangzhou, China, 2012. 385 – 393 (EI 收录,检索号20122515123980)

  \item Ma Y, Xu B, Bai Y, et al. Daily mood assessment based on mobile phone sensing. London,
United kingdom, 2012. 142 – 147. (EI 收录,检索号20122515131481)

  \item Ma Y, Xu B, Bai Y, et al. Infer Daily Mood using Mobile Phone Sensing. Ad Hoc \& Sensor Wireless Networks (已录用,SCI收录)

  \item Yin  Bai, Bin Xu, Yuanchao Ma,  et al.  Will you have a good sleep tonight? sleep quality prediction with mobile phone. The 7th International Conference on Body Area Networks (BodyNets 201 2). Oslo, Norway, September 2012.

  \item Bin  Xu,  Jian  Cui,  Guodong  Sun,  Yuanchao  Ma,  "Demo  Abstract:  Activity-aware  Heart  Sensing  in  Mobile
Healthcare", mHealthSys, in the 9th ACM Conference on Embedded Networked Sensor Systems (SenSys '11)

  \item Ma Y, Xu B, Yuan B, et al. User Behavior Analysis in Mobile Web. 19th ACM SIGKDD Conference on Knowledge Discovery and Data Mining (KDD) (在投)
  \end{enumerate}

  % \resumeitem{研究成果} % 有就写,没有就删除
  % \begin{enumerate}[{[}1{]}]
  % \item 任天令, 杨轶, 朱一平, 等. 硅基铁电微声学传感器畴极化区域控制和电极连接的
  %   方法: 中国, CN1602118A. (中国专利公开号.)
  % \item Ren T L, Yang Y, Zhu Y P, et al. Piezoelectric micro acoustic sensor
  %   based on ferroelectric materials: USA, No.11/215, 102. (美国发明专利申请号.)
  % \end{enumerate}
\end{resume}
