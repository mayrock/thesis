\chapter{总结及未来工作}
\label{cha:conclusion}

\section{论文总结}
利用海量的互联网记录数据和移动设备使用记录数据进行用户行为的分析和挖掘,是互联网及数据挖掘领域的重要课题,在面向个人和群体的多种服务上都有重要应用。本文利用移动环境下在设备端和运营商网络端产生的海量多种类数据,应用用户行为分析方法,针对个人用户的心境评估,以及群体用户的上网行为模式分析等问题提出了解决方法,并设计实现了完整的移动数据挖掘平台。

基于手机数据的个人心境评估方法中,首先给出了心境及心境评估问题的现状描述,并给出了运用移动环境下用户行为分析方法进行心境评估问题的上下文的形式化定义和描述;而后根据手机端可收集的用户行为相关的数据类型,定义了基于手机感知数据的用户行为特征,然后通过对实验数据的观察分析,结合心理学背景知识,使用概率因子图模型将社交关系、历史心境关系和用户行为特征统一建模,来评估用户心境。只使用历史心境和简单用户行为特征的模型,实验结果的评估精度达到了50\%;使用历史心境、所有用户行为特征和社交关系的完整模型的评估精度达到了70\%。

基于移动运营商数据的用户上网行为分析中,首先给出了所用的数据集规模、特征等概况描述,而后提出了基于用户与地理位置相关的上网行为,对地理位置进行功能区域划分的方法,并提出了将用户访问的主机地址聚类为网站的方法。在概率话题模型中,将地理功能区域、网站、用户等数据综合考虑,得出隐含的``用户兴趣''(网站类簇),从而对用户的移动互联网上网行为模式进行分析。实验中使用不同的评价指标计算了模型可靠度和还原(预测能力),结果表明,考虑到地理位置等附加因素的模型在各方面均有良好的表现,模型输出中发现了有意义的群体用户上网行为模式。

移动数据分析平台,包含了数据存储、处理、挖掘、可视化等完整的数据分析流程,实现了并行化的海量多来源异构数据的集成和挖掘。首先描述了平台定义的统一的Java语言编程接口,完整地支持对结构化和非结构化数据的查询、读取、写入等所有基本操作,实现了对异构的底层数据存储方式的封装。而后描述了平台的REST API,通过该标准接口实现了对远程调用的支持。而后描述了平台的数据可视化模块,支持针对数据集特征的不同,使用不同的可视化方法(地图可视化;统计可视化如饼图、折线图、柱状图等等)对其进行可配置的数据可视化。最后展示了平台的图形化管理界面。

\section{未来工作}

随着移动计算设备和移动互联网的飞速发展,移动环境下的用户行为分析定将持续成为热门的研究课题。下一步研究中可以考虑更多更丰富的用户行为数据,并且将用户行为分析方法应用到其他与行为相关的建模问题中。

应用手机数据进行心境评估方面,可以考虑更多的上下文信息,包括天气,以及用户所到过位置发生过的公共事件等。 在手机感知数据的处理方面,可以用不同来源的传感器数据定义整合的用户行为特征,从而改进模型的表现。

基于运营商数据进行用户上网行为分析方面,可以将用户间的社交关系考虑进模型。移动互联网中的上网行为可能通过社交网络进行传播,因此根据社交关系(通讯记录,同地同现等)来建立社交网络可以给每个用户更丰富的的上下文信息,从而改善模型的精确度。

移动数据分析平台方面,可以拓展平台支持的底层数据存储方式,同时支持更多的数据可视化方法和格式。同时在跨数据源的数据集成等方面从平台层面予以更多的支持。

